\documentclass{article}

\usepackage[utf8]{inputenc}
\usepackage{natbib}
\usepackage{graphicx}
\usepackage{xltabular}
\usepackage{tabularx}

\begin{document}

\begin{titlepage}
   \begin{center}
       \vspace*{1cm}

       {\LARGE Electronics and Computer Science\\
        Faculty of Engineering and Physical Sciences\\
        University of Southampton\\}


       \vspace{2.0cm}
        {\huge Cloud Application Development\\
        COMP3207\\}

            
       \vspace{1.0cm}
       {\huge Group Coursework\\
       BookIT\\}  
       
    
      \vspace{2.0cm}
      {\huge Team F} 
      
      
      \vspace{0.25cm}
      {\LARGE eas1g18 - Emilia Szynkowska\\
                ls9n17 - Liyu Su\\
                md8g17 - Manqi Dong\\
                mcn1g15 - Mihai Nuica\\
                tmf1g18 - Tomas Flynn\\
                dk3g18 - Dovydas David Kasinskas\\
                mhz1g18 - Milko Zlatev\\}
         
        
   \end{center}
\end{titlepage}

\section{Description of Prototype Functionality}
\subsection{Problem Background}
The Faculty of Engineering and Physical Sciences department at the University of Southampton is the biggest one out of the five that the institution has to offer. Of which, most belong to the ECS (Electronics and Computer Science). Due to its student population, it can be a difficult and arduous process to loan the required material from the University's library booking system during busier times for the said students.

\subsection{Proposed Solution}
In essence a dedicated library booking system for the specific field of ECS. The application our team would like to build is BookIT, an online library containing a database of books for technology and computer science. Here all the best and most popular books relating to this subject will be available. Our application allows users to borrow books from a distance, minimising the required physical contact.
\\\\
BookIT will allow users to create and log into an account and manage their details; they can also search the books database according to a range of criteria such as ID, name, author, edition, publisher, ISBN or ISSN number, or subject. Once the desired book is found, users will be able to borrow that book for a certain amount of time. The book will be marked as 'borrowed' in the database and will be associated with that user until it expires. The user may also be able to renew the book if they would like to use it for longer.

\subsection{Use Cases}
\begin{itemize}
  \item A student attempts to loan a book from the University's library system but finds that its slow and outdated and decides to use BookIT.
  \item A student registers on the BookIT application by entering the appropriate information and then logs in.
  \item The student decides to loan a book from the University's library using the application by searching for it using any of it's properties such such as ID, name, author, edition, publisher, ISBN or ISSN number, or subject.
  \item After 30 days the student can see that the book is overdue and decides to renew the book.
  \item The students decides to loan five more books on top of the renewed book but is notified of the five book limit.
  \Item After finishing with a book the student returns the book on the application.
\end{itemize}

\clearpage



\section{Tools \& Techniques}
\begin{tabularx}{\textwidth}{|l|X|}
\hline 
\multicolumn{2}{|c|}{Communication}\\
\hline
Messenger & Main tool for communication, mostly general, arranging meetings on other platforms, answering and asking questions and so on.\\
\hline
Microsoft Teams & Very little communication on this platform mainly to communicate between ourselves and lecturers.\\
\hline
Discord & This was the main communication platform for the actual development of the API here we posted links of the current work done, discussed what we're doing and will do, essentially planning.\\
\hline 
\hline
\multicolumn{2}{|c|}{Project Management}\\
\hline
Trello & Main organisation tool with lists and cards with labels that correspond to who does what. The cards can also marked as done, doing and To Do. \\
\hline
GitHub & The main and only version control system we used.\\
\hline 
\hline
\multicolumn{2}{|c|}{Design}\\
\hline
Google Docs & Where we wrote out the initial design ideas and some of the planning, also scripts for the reports.\\
\hline
Front End & Front end people enter ur shit here eaight\\
\hline 
\hline
\multicolumn{2}{|c|}{Development}\\
\hline
Google Cloud Platform & This is where we posted our functions for the back-end...\\
\hline
Python & The people responsible for the backed wrote the function in Python, PyCharm most likely and then deployed the functions on google cloud platform.\\
\hline
Other back end stuff & enter here\\
\hline 
\hline
\multicolumn{2}{|c|}{Testing \& QA}\\
\hline
Postman & Using the POST and GET requests to test the functions, to see what error codes are returned and if the data
manipulation was successful.\\
\hline
Google Cloud Platform& Google Cloud platform also had its own dedicated testing service which was similar to Postman, where a JSON file is passed and the function with a request is carried out.\\
\hline 


\end{tabularx}
\caption{\label{tab:table-name}Table 1: The table of Tools and Techniques used}
\clearpage



\section{Statistics}
\subsection{Code Lengths}
\begin{tabularx}{\textwidth}{|X|X|X|X|X|}
\hline 
Language & Files & Code & Comment & Blank\\
\hline
\hline
Python & 1 & 2 & 3 & 4\\
\hline
JavaScript & 1 & 2 & 3 & 4\\
\hline
Other & 1 & 2 & 3 & 4\\
\hline
\hline
Total & 1 & 2 & 3 & 4\\
\hline
\end{tabularx}
\caption{\label{tab:table-name}Table 2: Lengths of code}

\subsection{Something else}
Enter here whatever u feel is necessary
\clearpage



\section{Implementation}
\subsection{Back-end}
\subsection{Front-end}
\subsection{Testing}
\clearpage



\section{Evaluation}
\subsection{Collaboration, Functionality \& Code Quality}
\subsection{Strengths}
\subsection{Weaknesses}
\subsection{Improvements}
\clearpage


\end{document}
